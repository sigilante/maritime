\documentclass{article}
\usepackage{maritime}
\usepackage{longtable} % For multi-page tables
\usepackage{hyperref} % For hyperlinks
\usepackage{xcolor} % For color in examples if needed

\title{International Maritime Signal Flags Documentation}
\author{N.\ E.\ Davis, \texttt{@sigilante}, \texttt{\textasciitilde lagrev-nocfep}}
\date{\today}

\begin{document}

\maketitle

\section{Introduction}

The \textbf{maritime} package provides LaTeX commands for drawing international maritime signal flags using TikZ. This documentation covers the basic usage of the package.

\section{Installation}
To use this package, ensure your \texttt{maritime.sty} file is in a directory where LaTeX can find it (like \texttt{texmf-local/tex/latex/maritime/}) or in the same directory as your document.

\section{Usage}

\subsection{Drawing Flags}
Flags are drawn using specific commands:

\begin{itemize}
    \item \verb|\flagZ|: Draws flag Z.
    \item \verb|\flagO|: Draws flag O.
    \item \verb|\flagR|: Draws flag R.
    \item \verb|\flagP|: Draws flag P.
\end{itemize}

Here's how to use them.  We suggest using \verb|\quad| to separate the flags as otherwise they abut.

\begin{verbatim}
  \flagZ
  \flagO
  \flagR
  \flagP
\end{verbatim}

Example:

\flagZ \quad \flagO \quad \flagR \quad \flagP

Furthermore, we provide the standard name at \verb|\flagXname| and the blazon at \verb|\flagXblazon| for each flag \verb|X|.

Example:

\begin{verbatim}
  \flagZname: \flagZblazon
  \flagOname: \flagOblazon
  \flagRname: \flagRblazon
  \flagPname: \flagPblazon
\end{verbatim}

\noindent
\flagZname: \flagZblazon \\
\flagOname: \flagOblazon \\
\flagRname: \flagRblazon \\
\flagPname: \flagPblazon

\section{The Flags}

We supply the alphabetical International Code of Symbols maritime flags and the NATO number flags.

\begin{longtable}{lcl}
  \caption{International Maritime Signal Flags, Letters} \\
  A \flagAname & \flagA & \flagAblazon \\
  B \flagBname & \flagB & \flagBblazon \\
  C \flagCname & \flagC & \flagCblazon \\
  D \flagDname & \flagD & \flagDblazon \\
  E \flagEname & \flagE & \flagEblazon \\
  F \flagFname & \flagF & \flagFblazon \\
  G \flagGname & \flagG & \flagGblazon \\
  H \flagHname & \flagH & \flagHblazon \\
  I \flagIname & \flagI & \flagIblazon \\
  J \flagJname & \flagJ & \flagJblazon \\
  K \flagKname & \flagK & \flagKblazon \\
  L \flagLname & \flagL & \flagLblazon \\
  M \flagMname & \flagM & \flagMblazon \\
  N \flagNname & \flagN & \flagNblazon \\
  O \flagOname & \flagO & \flagOblazon \\
  P \flagPname & \flagP & \flagPblazon \\
  Q \flagQname & \flagQ & \flagQblazon \\
  R \flagRname & \flagR & \flagRblazon \\
  S \flagSname & \flagS & \flagSblazon \\
  T \flagTname & \flagT & \flagTblazon \\
  U \flagUname & \flagU & \flagUblazon \\
  V \flagVname & \flagV & \flagVblazon \\
  W \flagWname & \flagW & \flagWblazon \\
  X \flagXname & \flagX & \flagXblazon \\
  Y \flagYname & \flagY & \flagYblazon \\
  Z \flagZname & \flagZ & \flagZblazon \\
\end{longtable}

\begin{longtable}{lcl}
  \caption{International Maritime Signal Flags, Numbers} \\
  0 \flagZeroName & \flagZero & \flagZeroBlazon \\
  1 \flagOneName & \flagOne & \flagOneBlazon \\
  2 \flagTwoName & \flagTwo & \flagTwoBlazon \\
  3 \flagThreeName & \flagThree & \flagThreeBlazon \\
  4 \flagFourName & \flagFour & \flagFourBlazon \\
  5 \flagFiveName & \flagFive & \flagFiveBlazon \\
  6 \flagSixName & \flagSix & \flagSixBlazon \\
  7 \flagSevenName & \flagSeven & \flagSevenBlazon \\
  8 \flagEightName & \flagEight & \flagEightBlazon \\
  9 \flagNineName & \flagNine & \flagNineBlazon \\
\end{longtable}

\subsection{Customization}

\subsubsection{Size}

The standard flag size is \verb|15ex|, meaning the height of the flag is 15 times the height of the letter `x'.

\begin{verbatim}
  \flagSix
  \flagNine
\end{verbatim}

Example:

\flagSix \quad \flagNine

Flags can be scaled using standard LaTeX font size commands:

\begin{verbatim}
{\Large \flagU}
{\normalsize \flagR}
{\tiny \flagB}
\end{verbatim}

Example:

{\Large \flagU} \quad {\normalsize \flagR} \quad {\tiny \flagB}

They may also be explicitly scaled using the \verb|\resizebox| command:

\begin{verbatim}
  \resizebox{!}{5cm}{\flagI}
  \resizebox{!}{0.5cm}{\flagT}
\end{verbatim}

Example:

\resizebox{!}{5cm}{\flagI} \quad \resizebox{!}{0.5cm}{\flagT}

\section{Conclusion}

% enjoy yourself
\begin{center}
\flagE \space \flagN \space \flagJ \space \flagO \space \flagY \\

\flagY \quad \flagO \quad \flagU \quad \flagR \\
\quad \flagS \quad \flagE \quad \flagL \quad \flagF
\end{center}

\end{document}
